\documentclass[10pt]{beamer} % Mniejsza czcionka bazowa może być lepsza przy dużej liczbie slajdów

\usetheme{metropolis}       % Nowoczesny motyw
% \usetheme{Frankfurt}      % Inna popularna alternatywa, jeśli metropolis nie działa
% \usetheme{CambridgeUS}    % Klasyczny, akademicki
% \usecolortheme{whale}      % Domyślny dla metropolis, można zmienić
\usepackage[trueids]{grffile} % Nowsza opcja
\usepackage[utf8]{inputenc}
\usepackage[T1]{fontenc}
\usepackage{lmodern}          % Lepsze czcionki
\usepackage[polish]{babel}    % Język polski
\usepackage{graphicx}         % Do włączania obrazków
\usepackage{booktabs}         % Do ładnych tabel (jeśli będziesz dodawał)
\usepackage{amsmath}
\usepackage{amsfonts}
\usepackage{amssymb}
\usepackage{hyperref}         % Linki klikalne
\usepackage{multicol}         % Do układania obrazków w kolumnach

% --- Ustawienia dla Metropolis ---
% \metroset{block=fill}       % Wypełnione bloki (theorem, example)
\metroset{numbering=fraction} % Numeracja slajdów jako X/Y
% \metroset{progressbar=frametitle} % Pasek postępu pod tytułem slajdu
% \metroset{sectionpage=none}  % Wyłączenie osobnych stron dla sekcji (oszczędność miejsca)

% Definicja kolorów (opcjonalnie)
% \definecolor{UGreen}{HTML}{007A5E} % Przykład koloru
% \setbeamercolor{palette primary}{bg=UGreen,fg=white}

\title{Zintegrowana Analiza Danych Wielotematycznych}
\subtitle{Wizualizacje z Różnych Zbiorów Danych}
\author{Bartek Grabowski \newline Witold Winiarski} % Zmień na swoje imię i nazwisko
\date{\today}
% \institute{Twoja Instytucja/Uczelnia (opcjonalnie)}
% \logo{\includegraphics[height=1cm]{logo.png}} % Jeśli masz logo

% --- Katalog z wykresami ---
% Załóżmy, że wykresy są w podkatalogach względem pliku .tex
\graphicspath{{plots\_trump/}{plots\_bieguny/}{plots\_speedcubing/}{plots\_slackline/}}

% --- Makra ułatwiające wstawianie wykresów ---
% Wykres na cały slajd
% Poprawiona definicja makra
\newcommand{\fullslideimage}[3][width=\textwidth,height=0.75\textheight,keepaspectratio]{
  \begin{frame}{#3} % Użyj #3 jako tytułu slajdu
    \begin{figure}
      \centering
      \includegraphics[#1]{#2} % Użyj #2 jako nazwy pliku
    \end{figure}
  \end{frame}
}

% Dwa wykresy obok siebie
\newcommand{\twoslideimages}[4][width=0.48\textwidth,height=0.65\textheight,keepaspectratio]{%
  \begin{frame}{#4}
    \begin{columns}[T] % T = top align
      \begin{column}{0.5\textwidth}
        \begin{figure}
          \centering
          \includegraphics[#1]{#2}
          % \caption{Podpis pod pierwszym obrazkiem (opcjonalnie)}
        \end{figure}
      \end{column}
      \begin{column}{0.5\textwidth}
        \begin{figure}
          \centering
          \includegraphics[#1]{#3}
          % \caption{Podpis pod drugim obrazkiem (opcjonalnie)}
        \end{figure}
      \end{column}
    \end{columns}
  \end{frame}
}

% Dwa wykresy jeden pod drugim
\newcommand{\twoimagesstacked}[4][width=0.8\textwidth,height=0.35\textheight,keepaspectratio]{%
  \begin{frame}{#4}
    \centering
    \includegraphics[#1]{#2}\\[1ex] % Pierwszy obrazek
    \includegraphics[#1]{#3}        % Drugi obrazek
  \end{frame}
}


\begin{document}

% --- Strona Tytułowa ---
\maketitle

% --- Spis Treści ---
\begin{frame}{Spis Treści}
  \setbeamertemplate{section in toc}[sections numbered]
  \setbeamertemplate{subsection in toc}[subsections numbered]
  \tableofcontents
\end{frame}

% --- Sekcja 1: Analiza Tweetów Trumpa ---
\section{Analiza Tweetów Donalda Trumpa}

\begin{frame}{Wprowadzenie do Analizy Tweetów}
  \begin{itemize}
    \item Analiza zbioru tweetów Donalda Trumpa.
    \item Cel: Zrozumienie wzorców aktywności, głównych tematów i sentymentu.
    \item Wykorzystane narzędzia: R, pakiety tidyverse, tidytext, ggplot2.
  \end{itemize}
\end{frame}

\subsection{Aktywność i Charakterystyka Treści}
\fullslideimage{liczba\_tweetow\_miesiecznie.png}{Aktywność Tweetowania w Czasie}
\fullslideimage{rozklad\_dlugosci\_tweetow.png}{Rozkład Długości Tweetów}
\fullslideimage{chmuraslow.png}{Chmura Najczęściej Występujących Słów} % Załóżmy, że zapisałeś ją jako PNG
\fullslideimage{top20\_slow.png}{Top 20 Najczęstszych Słów}

\subsection{Hashtagi, Wzmianki i Bigramy}
\fullslideimage{top20\_hashtagi.png}{Najpopularniejsze Hashtagi}
\fullslideimage{top20\_wzmianki.png}{Najczęściej Wzmiankowani Użytkownicy}
\fullslideimage{top20\_bigramy.png}{Najczęstsze Pary Słów (Bigramy)}

\subsection{Analiza Sentymentu}
\fullslideimage{analiza\_sentymentu\_bing.png}{Ogólna Analiza Sentymentu (Bing)}


% --- Sekcja 2: Wędrówka Biegunów Geomagnetycznych ---
\section{Wędrówka Biegunów Geomagnetycznych}

\begin{frame}{Wprowadzenie do Biegunów Geomagnetycznych}
  \begin{itemize}
    \item Analiza danych dotyczących położenia północnego bieguna geomagnetycznego (model IGRF).
    \item Analiza danych deklinacji magnetycznej z regionu GRSM\@.
    \item Cel: Wizualizacja zmian położenia bieguna oraz charakterystyki pola magnetycznego.
  \end{itemize}
\end{frame}

\subsection{Północny Biegun Geomagnetyczny (IGRF)}
\fullslideimage{igrf\_north\_pole\_wander\_map\_labels.png}{Wędrówka Północnego Bieguna Geomagnetycznego (IGRF)}
\twoimagesstacked{igrf\_north\_pole\_latitude\_vs\_time.png}{igrf\_north\_pole\_longitude\_vs\_time.png}{Zmiany Położenia Północnego Bieguna (IGRF) w Czasie}

\subsection{Dane Deklinacji Magnetycznej (GRSM)}
\fullslideimage{grsm\_declination\_map.png}{Mapa Deklinacji Magnetycznej (GRSM)}
\twoslideimages{grsm\_declination\_histogram.png}{grsm\_annual\_drift\_histogram.png}{Rozkład Deklinacji i Rocznego Dryftu (GRSM)}
\twoimagesstacked{grsm\_declination\_vs\_latitude.png}{grsm\_declination\_vs\_longitude.png}{Zależność Deklinacji od Położenia (GRSM)}


% --- Sekcja 3: Analiza Wyników Zawodów (Speedcubing) ---
\section{Analiza Wyników Zawodów Speedcubingowych}
% Nazwy plików PNG dla tej sekcji są z Twojego skryptu "speedcubing"

\begin{frame}{Wprowadzenie do Analizy Zawodów Speedcubingowych}
    \begin{itemize}
        \item Analiza danych dotyczących najlepszych wyników średnich i pojedynczych oraz ogólnych informacji o zawodach.
        \item Cel: Identyfikacja trendów, najlepszych zawodników i dominujących krajów.
    \end{itemize}
\end{frame}

\subsection{Rozkład Czasów i Dominacja Krajów}
\fullslideimage{02b\_rozkłady\_czasow\_łączone.png}{Rozkłady Najlepszych Czasów}
\fullslideimage{04b\_top10\_krajow\_łączone.png}{Top 10 Krajów wg Liczby Zawodników}
\fullslideimage{15\_mapa\_zawodnicy\_srednie.png}{Mapa Świata: Liczba Zawodników w Rankingu Średnich}

\subsection{Wyniki Najlepszych Zawodników}
\fullslideimage{06b\_czas\_vs\_ranking\_łączone.png}{Najlepsze Czasy vs Ranking (Top 5 Krajów)}
\fullslideimage{10\_top10\_zawodnikow\_srednie.png}{Top 10 Zawodników (Najlepsze Średnie)}
\fullslideimage{09\_porownanie\_czasow\_pojed\_sred.png}{Porównanie Czasów Pojedynczych i Średnich}
\fullslideimage{14\_boxplot\_srednie\_top5\_kraje.png}{Rozkład Czasów Średnich dla Top 5 Krajów}


\subsection{Charakterystyka Zawodów}
\fullslideimage{07\_liczba\_zawodow\_lata.png}{Liczba Zawodów na Przestrzeni Lat}
\fullslideimage{08\_top15\_kraje\_organizujace.png}{Kraje Najczęściej Organizujące Zawody}
\fullslideimage{12b\_rekordy\_rok\_łączone.png}{Liczba Ustanowionych Rekordów wg Roku}
\fullslideimage{13\_top10\_zawody\_rekordy\_srednie.png}{Zawody z Największą Liczbą Rekordów Średnich}
\fullslideimage{17\_unikalne\_kraje\_rekordy\_srednie\_lata.png}{Liczba Unikalnych Krajów w Rankingu Średnich (wg Roku Zawodów)}


% --- Sekcja 4: Analiza Incydentów na Slackline ---
\section{Analiza Incydentów na Slackline}
% Nazwy plików PNG dla tej sekcji są z Twojego skryptu "slackline"

\begin{frame}{Wprowadzenie do Analizy Incydentów na Slackline}
    \begin{itemize}
        \item Analiza zgłoszonych incydentów związanych z uprawianiem slackline.
        \item Cel: Zrozumienie częstotliwości, rodzajów urazów, typów incydentów i czynników ryzyka.
    \end{itemize}
\end{frame}

\subsection{Trendy i Rodzaje Urazów}
\fullslideimage{01\_incydenty\_lata.png}{Liczba Zgłoszonych Incydentów wg Roku}
\fullslideimage{02\_top10\_rodzaje\_urazow.png}{Top 10 Najczęstszych Rodzajów Urazów}
\fullslideimage{03\_top10\_lokalizacje\_urazow.png}{Top 10 Najczęstszych Lokalizacji Urazów}
\fullslideimage{07\_odsetek\_urazy\_vs\_bez.png}{Odsetek Incydentów z Urazami vs Bez Urazów}

\subsection{Typy Incydentów i Slackline'ów}
\fullslideimage{04\_top10\_typy\_incydentow.png}{Top 10 Najczęstszych Typów Incydentów}
\fullslideimage{05\_incydenty\_rodzaj\_slackline.png}{Liczba Incydentów wg Rodzaju Slackline'a}
\fullslideimage{08\_urazy\_wg\_typu\_slackline.png}{Najczęstsze Urazy wg Typu Slackline'a}
\fullslideimage{12\_injury\_vs\_nearmiss\_typ\_slackline.png}{Porównanie `Injury' vs `Near Miss' wg Typu Slackline'a}

\subsection{Geografia i Analiza Szczegółowa}
\fullslideimage{06\_top10\_kraje\_incydenty.png}{Top 10 Krajów z Największą Liczbą Incydentów}
\fullslideimage{10\_mapa\_incydenty\_kraje.png}{Mapa Świata: Liczba Incydentów wg Kraju}
\fullslideimage{14\_heatmap\_uraz\_lokalizacja.png}{Heatmapa: Rodzaj Urazu vs Lokalizacja}
% Wykres Alluvial może być duży, rozważ prezentację jego fragmentu lub opisu
\fullslideimage{13\_alluvial\_slack\_uraz\_lok.png}{Przepływ: Typ Slackline -> Rodzaj Urazu -> Lokalizacja}
% Analiza przypadków śmiertelnych
\begin{frame}{Analiza Przypadków Śmiertelnych na Slackline}
    % Możesz tu wstawić mniejsze obrazki, jeśli masz dla nich miejsce, np. 15a, 15b, 15c
    \includegraphics[width=0.45\textwidth]{15a\_death\_wg\_kraju.png}
    \includegraphics[width=0.45\textwidth]{15b\_death\_wg\_typu\_slackline.png}
\end{frame}

% --- Podsumowanie (opcjonalnie) ---
\section*{Podsumowanie} % Użyj \section* jeśli nie chcesz numeracji
\begin{frame}{Podsumowanie i Główne Wnioski (Część 1)}
  \begin{itemize}
    \item Przedstawiona zintegrowana analiza danych wielotematycznych, obejmująca tweety Donalda Trumpa, wędrówkę biegunów geomagnetycznych, wyniki zawodów speedcubingowych oraz incydenty na slackline, dostarczyła szeregu interesujących spostrzeżeń.

    \item \textbf{Analiza Tweetów Donalda Trumpa:}
    \begin{itemize}
        \item Zaobserwowano zmienną aktywność tweetowania w czasie oraz charakterystyczny rozkład długości treści.
        \item Identyfikacja kluczowych tematów, słów, hashtagów i wzmiankowanych użytkowników pozwoliła zrozumieć główne narracje.
        \item Analiza sentymentu (wg Bing) wykazała przewagę treści pozytywnych.
    \end{itemize}

    \item \textbf{Wędrówka Biegunów Geomagnetycznych:}
    \begin{itemize}
        \item Model IGRF ukazał dynamiczną wędrówkę północnego bieguna geomagnetycznego w okresie 1590--2025.
        \item Dane z regionu GRSM zobrazowały przestrzenne zróżnicowanie deklinacji magnetycznej i jej zależność od lokalizacji.
    \end{itemize}
  \end{itemize}
\end{frame}

\begin{frame}{Podsumowanie i Główne Wnioski (Część 2)}
  \begin{itemize}
    \item \textbf{Analiza Wyników Zawodów Speedcubingowych:}
    \begin{itemize}
        \item Rozkłady czasów wskazują na wysoki poziom rywalizacji; zidentyfikowano dominujące kraje i najlepszych zawodników.
        \item Zaobserwowano dynamiczny wzrost liczby organizowanych zawodów na przestrzeni lat.
    \end{itemize}

    \item \textbf{Analiza Incydentów na Slackline:}
    \begin{itemize}
        \item Zgłoszenia incydentów wykazywały tendencję wzrostową; zidentyfikowano najczęstsze rodzaje i lokalizacje urazów.
        \item Analiza typów incydentów, rodzajów slackline'ów oraz danych geograficznych wskazała na obszary szczególnego ryzyka.
        \item Większość zgłoszonych incydentów wiązała się z urazem.
    \end{itemize}

    \item \textbf{Ogólne Spostrzeżenia:}
    \begin{itemize}
        \item Wizualizacja danych okazała się kluczowym narzędziem w odkrywaniu wzorców i trendów.
        \item Każdy z analizowanych zbiorów danych oferuje potencjał do dalszych, pogłębionych badań.
    \end{itemize}
  \end{itemize}
\end{frame}

% --- Strona Końcowa ---
\begin{frame}
  \centering
  \Huge Dziękuję za uwagę!
  \\[2em] % Większy odstęp
  {\large Pytania?}
\end{frame}

\end{document}